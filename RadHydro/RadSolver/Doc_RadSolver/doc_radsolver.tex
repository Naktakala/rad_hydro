\documentclass[10pt,letterpaper,notitlepage]{article}
\usepackage[utf8]{inputenc}
\usepackage{amsmath}
\usepackage{amsfonts}
\usepackage{amssymb}
\usepackage{graphicx}
\usepackage{cancel}
\usepackage{float}
\usepackage{cite}
\usepackage{fancyvrb}

\usepackage[ruled,vlined]{algorithm2e}


\usepackage[left=0.75in, right=0.75in, bottom=1.0in,top=0.75in]{geometry}

%\usepackage{caption} 
%\captionsetup[table]{skip=10pt}
%\usepackage[font=small,labelfont=bf]{caption}

\usepackage{comment}
\usepackage{listings}

\usepackage{color}
\definecolor{Brown}{cmyk}{0,0.81,1,0.60}
\definecolor{OliveGreen}{cmyk}{0.64,0,0.95,0.40}
\definecolor{CadetBlue}{cmyk}{0.62,0.57,0.23,0}

\definecolor{ColorLevel1}{rgb}{0.000,0.125,0.875}
\definecolor{ColorLevel2}{rgb}{0.000,0.375,0.625}
\definecolor{ColorLevel3}{rgb}{0.000,0.625,0.375}
\definecolor{ColorLevel4}{rgb}{0.000,0.875,0.125}
\definecolor{ColorLevel5}{rgb}{0.125,0.875,0.000}
\definecolor{ColorLevel6}{rgb}{0.375,0.625,0.000}
\definecolor{ColorLevel7}{rgb}{0.625,0.375,0.000}
\definecolor{ColorLevel8}{rgb}{0.875,0.125,0.000}

\usepackage{multicol}

\usepackage{appendix}

\usepackage{fancyhdr}
%\usepackage[colorlinks=true,linkcolor=blue,urlcolor=black,bookmarksopen=true,bookmarks]{hyperref}
\usepackage{bookmark}

\usepackage{multicol}
\setlength{\columnseprule}{1pt}

\numberwithin{equation}{section} 

\setcounter{tocdepth}{4}

%============================= Put document title here
\newcommand{\DOCTITLE}{Radiative heat transfer solver with fluid motion}  

%=============================  Load list of user-defined commands
% Mark URL's
\newcommand{\URL}[1]{{\textcolor{blue}{#1}}}
%
% Ways of grouping things
%
\newcommand{\bracket}[1]{\left[ #1 \right]}
\newcommand{\bracet}[1]{\left\{ #1 \right\}}
\newcommand{\fn}[1]{\left( #1 \right)}
\newcommand{\ave}[1]{\left\langle #1 \right\rangle}
\newcommand{\norm}[1]{\Arrowvert #1 \Arrowvert}
\newcommand{\abs}[1]{\arrowvert #1 \arrowvert}
%
% Partial derivative
\newcommand{\partialderiv}[2]{\frac{\partial #1}{\partial #2}}
%
% Bold quantities
% 
\newcommand{\Omegabf}{\mathbf{\Omega}}
\newcommand{\bnabla}{\boldsymbol{\nabla}}
\newcommand{\position}{\mathbf{x}}
\newcommand{\velocity}{\mathbf{u}}
\newcommand{\dotp}{\boldsymbol{\cdot}}

\newcommand{\RadE}{\mathcal{E}}
\newcommand{\RadF}{\boldsymbol{\mathcal{F}}}
\newcommand{\RadP}{\boldsymbol{\mathcal{P}}}
\newcommand{\HydroF}{\mathcal{F}^H}
\newcommand{\HydroU}{\mathbf{U}}
\newcommand{\RadJ}{\mathbf{J}}

\newcommand{\SumOverCellFaces}{\sum_f^{N_{f,c}{-1}}}
\newcommand{\AreaVec}{\mathbf{A}}

\newcommand{\uvec}[1]{\boldsymbol{\hat{\textbf{#1}}}}

\newcommand{\ihat}{\uvec{\i}}
\newcommand{\jhat}{\uvec{\j}}
\newcommand{\khat}{\uvec{k}}

\newcommand{\hatbf}[1]{\hat{\mathbf{#1}}}

\newcommand{\half}{\frac{1}{2}}

%\newcommand{\ihat}{\boldsymbol{\hat{\textbf{\i}}}}
%\newcommand{\jhat}{\boldsymbol{\hat{\textbf{\j}}}}
%\newcommand{\khat}{\boldsymbol{\hat{\textbf{\k}}}}

%\newcommand{\ihat}{{\bm{\hat{\textnormal{\bfseries\i}}}}}
%\newcommand{\jhat}{{\bm{\hat{\textnormal{\bfseries\j}}}}}
%\newcommand{\khat}{{\bm{\hat{\textnormal{\bfseries\k}}}}}
%
% Vector forms
%
\renewcommand{\vec}[1]{\mbox{$\stackrel{\longrightarrow}{#1}$}}
\renewcommand{\div}{\mbox{$\vec{\mathbf{\nabla}} \cdot$}}
\newcommand{\grad}{\mbox{$\vec{\mathbf{\nabla}}$}}
\newcommand{\bb}[1]{\bar{\bar{#1}}}
%
% Vector forms boldfaced
\newcommand{\bvec}[1]{\mathbf{#1}}
\newcommand{\bdiv}{\boldsymbol{\nabla} \boldsymbol{\cdot}}
\newcommand{\bgrad}{\bnabla}
\newcommand{\mat}[1]{\bar{\bar{#1}}}
%
%
% Equation beginnings and endings
%
% Un-numbered equation with alignment
\newcommand{\beq}{\begin{equation*} \begin{aligned}}
\newcommand{\eeq}{\end{aligned}\end{equation*}}
% Numbered equation with alignment
\newcommand{\beqn}{\begin{equation}\begin{aligned}}
\newcommand{\eeqn}{\end{aligned}\end{equation}}  

%
% Quick commands for symbols
%
\newcommand{\Edensity}{\mathcal{E}}


\newcommand{\jcr}[1]{\textcolor{magenta}{#1}}
\usepackage[normalem]{ulem}
\newcommand{\ssout}[1]{\sout{\textcolor{magenta}{#1}}}

%
% Code syntax highlighting
%
%\lstset{language=C++,frame=ltrb,framesep=2pt,basicstyle=\linespread{0.8} \small,
%	keywordstyle=\ttfamily\color{OliveGreen},
%	identifierstyle=\ttfamily\color{CadetBlue}\bfseries,
%	commentstyle=\color{Brown},
%	stringstyle=\ttfamily,
%	showstringspaces=true,
%	tabsize=2,}

\lstset{language=C++,frame=ltrb,framesep=8pt,basicstyle=\linespread{0.8} \Large,
commentstyle=\ttfamily\color{OliveGreen},
keywordstyle=\ttfamily\color{blue},
identifierstyle=\ttfamily\color{CadetBlue}\bfseries,
stringstyle=\ttfamily,
tabsize=2,
showstringspaces=false,
numbers=left,
captionpos=t}

\renewcommand{\lstlistingname}{\textbf{Code Snippet}}% Listing -> Code Snippet


\begin{document}
\noindent
{\LARGE\textbf{\DOCTITLE}}
\newline
\newline
\newline
\noindent
{\Large Jan I.C. Vermaak$^{1,2}$, Jim E. Morel$^{1,2}$}
\newline
\noindent\rule{\textwidth}{1pt}
{\small $^1$Center for Large Scale Scientific Simulations, Texas A\&M Engineering Experiment Station, College Station, Texas, USA.}
\newline\noindent
{\small $^2$Nuclear Engineering Department, Texas A\&M University, College Station, Texas, USA.}
\newline
\newline
\textbf{Abstract:}\newline\noindent
Work is work for some, but for some it is play.
\newline
\newline\noindent
{\small
\textbf{Keywords:} hydrodynamics}

\tableofcontents

\section{Definitions}
\subsection{Independent variables}
We refer to the following independent variables:
\begin{itemize}
	\item Position in the cartesian space $\{x,y,z\}$ is denoted with $\position$ and each component having units $[cm]$.
	\item Direction, $\{\varphi, \theta\}$, is denoted with $\Omegabf$ which takes on the form 
	$$
	\Omegabf = 
	\begin{bmatrix}
		\Omega_x \\ \Omega_y \\ \Omega_z
	\end{bmatrix}
	\text{ and/or }
	\Omegabf = 
	\begin{bmatrix}
		\sin\theta \cos\varphi \\ \sin\theta \sin\varphi \\ \cos\theta
	\end{bmatrix},
	$$
	where $\varphi$ is the azimuthal-angle and $\theta$ is the polar-angle, both in spherical coordinates. Commonly, $\cos\theta$, is denoted with $\mu$. The general dimension of angular phase space is $[steridian]$.
	\item Photon frequency, $\nu$ in $[Hertz]$ or $[s^{-1}]$.
	\item Time, $t$ in $[s]$.
\end{itemize} 

\vspace{0.5cm}
\subsection{Dependent variables}
We use the following basic dependent variables:
\begin{itemize}
\item The foundation of the dependent unknowns is the \textbf{radiation angular intensity}, $I(\position, \Omegabf, \nu,t)$ with units $[Joule/cm^2 {-} s {-} steradian {-} Hz]$. We often use the corresponding angle-integral of this quantity, $\phi(\position,\nu,t)$, and define it as
	\beqn 
	\phi(\position,\nu,t) = \RadE c = \int_{4\pi} I(\position,\Omegabf,\nu,t) \ d\Omegabf
	\eeqn 
	with units $[Joule/cm^2 {-} s {-} Hz]$. Where $c$ is the speed of light.
\item  The \textbf{radiation energy density}, $\Edensity$, is 
	\beqn 
	\Edensity(\position, \nu, t) = \frac{\phi}{c}  = 
	\frac{1}{c} \int_{4\pi} I(\position,\Omegabf,\nu,t) \ d\Omegabf
	\eeqn 
	with units $[Joule/cm^3 {-} Hz]$.
\item The \textbf{radiation energy flux}, $\RadF$, is
	\beqn 
    \RadF(\position, \nu, t) = \int_{4\pi} \Omegabf \  I(\position, \Omegabf, \nu, t) d\Omegabf
	\eeqn 
\item \textbf{Radiation pressure}, $\RadP$, is 
	\beqn
	\RadP(\position, \nu, t) = \frac{1}{c}\int_{4\pi} \Omegabf \otimes \Omegabf I(\position, \Omegabf, \nu, t) d\Omegabf
	\eeqn 
	and is a tensor.
\end{itemize}


\vspace{0.5cm}
\subsection{Blackbody radiation}
A blackbody radiation source, $B(\nu,T)$, is properly described by \textbf{Planck's law},
\beqn \label{eq:plancks_law}
B(\nu,T) = \frac{2h\nu^3}{c^2} \frac{1}{e^{\frac{h\nu}{k_B T}} - 1  }
\eeqn 
with units $[Joule/cm^2 {-} s {-} steridian-Hz]$ where $h$ is Planck's constant and $k_B$ is the Boltzmann constant.

If we integrate the blackbody source over all angle-space and frequencies then we get the mean radiation intensity from a blackbody at temperature $T$ as
\beqn 
\int_0^\infty \int_{4\pi}  B(\nu, T) \ d\Omegabf d\nu
&=\int_0^\infty \int_{4\pi}  \frac{2h\nu^3}{c^2} \frac{1}{e^{\frac{h\nu}{k_B T}} - 1  } \ d\Omegabf d\nu \\
&=4\pi \int_0^\infty \frac{2h\nu^3}{c^2} \frac{1}{e^{\frac{h\nu}{k_B T}} - 1  } \  d\nu \\
&= a c T^4,
\eeqn 
with units $[Joule/cm^2 {-} s {-} steridian]$ and where $a$ is the \textbf{blackbody radiation constant} given by
\beqn 
a = \frac{8\pi^5 k_B^4}{15 h^3 c^3}.
\eeqn 
\newline
\newline
In both cases this unfortunately is only the intensity. Following Kirchoff's law, which states that the emission and absorption of radiation must be equal in equilibrium, we can determine the \textbf{blackbody emission rate}, $S_{bb}$, from the absorption rate as
\beqn 
S_{bb}(\nu, T) = \rho \kappa(\nu) B(\nu, T),
\eeqn 
with units $[Joule/cm^3 {-} s {-} steridian {-} Hz]$
where $\rho$ is the material density $[g/cm^3]$ and $\kappa$ is the opacity $[cm^2/g]$. The combination $\rho \kappa$ is also equal to the macroscopic absorption cross section $\sigma_a$, therefore $\rho \kappa(\nu) = \sigma_a$. Data for the opacity of a material is normally available in the form of either the \textbf{Rosseland opacity}, $\kappa_{Rs}$, or the \textbf{Planck opacity}, $\kappa_{Pl}$.






\newpage
\section{Conservation equations}
\subsection{Conservation equation - Radiative transfer}
The basic statement of conservation, is
\beqn 
\frac{1}{c} \frac{\partial I(\position, \Omegabf, \nu, t)}{\partial t} &=
-\Omegabf \dotp \bnabla I(\position, \Omegabf, \nu, t)
- \sigma_t(\position,\nu) I(\position, \Omegabf, \nu, t) \\
&+ \int_0^{\infty} \int_{4\pi} \frac{\nu}{\nu'} \sigma_s(\position,\nu'{\to}\nu,\Omegabf'{\dotp}\Omegabf) I(\position, \Omegabf', \nu', t)  d\nu' d\Omegabf' \\
&+ \sigma_a(\position,\nu) B(\nu,T(\position, t))+S
\eeqn 
where $S$ is any other sources/sinks of radiation intensity. 

\subsection{Radiative transfer assuming isotropic Thompson scattering}
Assuming Thomson-scattering\footnote{Thomson scattering is the elastic scattering of electromagnetic radiation by a free charged particle. The particle's kinetic energy- as well as the photon's frequency, does not change in such a scattering. The scattering is also isotropic.} is the only form of scattering, gives
\beqn 
\frac{1}{c} \frac{\partial I(\position, \Omegabf, \nu, t)}{\partial t} &=
-\Omegabf \dotp \bnabla I(\position, \Omegabf, \nu, t)
- \sigma_t(\position,\nu) I(\position, \Omegabf, \nu, t) \\
&+ \frac{\sigma_s(\position,\nu)}{4\pi} c \RadE(\position, \nu)
+ \sigma_a(\position,\nu) B(\nu,T(\position, t))+S
\eeqn 
where $S$ is any other sources/sinks of radiation intensity. 
\newline
\newline
\textbf{Using energy instead of frequency, $\nu\to E$:}
\beqn 
\frac{1}{c} \frac{\partial I(\position, \Omegabf, E, t)}{\partial t} &=
-\Omegabf \dotp \bnabla I(\position, \Omegabf, E, t)
- \sigma_t(\position,E) I(\position, \Omegabf, E, t) \\
&+ \frac{\sigma_s(\position,E)}{4\pi} c \RadE(\position, E)
+ \sigma_a(\position,E) B(E,T(\position, t))+S
\eeqn 
where $S$ is any other sources/sinks of radiation intensity. 

\subsection{Radiative transfer with material motion corrections}
Applying relativistic corrections for a material in motion, we can derive 
\beqn 
\frac{1}{c} \frac{\partial I(\position, \Omegabf, E, t)}{\partial t} &=
-\Omegabf \dotp \bnabla I(\position, \Omegabf, E, t)
- \biggr(\frac{E_0}{E}\biggr)\sigma_t(\position,E_0) I(\position, \Omegabf, E, t) \\
&+ \biggr(\frac{E}{E_0}\biggr)^2\frac{\sigma_s(\position,E)}{4\pi} \int_{4\pi} \biggr(\frac{E_0}{E'}\biggr) I(\position, \Omegabf', E', t) d\Omegabf'
+ \biggr(\frac{E}{E_0}\biggr)^2 \sigma_a(\position,E_0) B(E_0,T(\position, t))+S,
\eeqn 
where
\beqn 
E_0 = E \gamma \biggr(1-\Omegabf \dotp \frac{\mathbf{u}}{c}\biggr)
\eeqn 
\beqn 
\gamma = \biggr[ 1-\biggr(\dfrac{||\mathbf{u}||}{c}\biggr)^2 \biggr]^{-\frac{1}{2}}
\eeqn 
\beqn 
\frac{E_0}{E'} = \gamma \biggr(1-\Omegabf' \dotp \frac{\mathbf{u}}{c} \biggr)
\eeqn 
\beqn 
E' = E \dfrac{1-\Omegabf \dotp \dfrac{\mathbf{u}}{c}}{1-\Omegabf' \dotp \dfrac{\mathbf{u}}{c}}
\eeqn 

\subsection{Radiative transfer with material velocity dependencies expanded to $\mathcal{O}(v/c)$}
\beqn 
&\frac{1}{c} \frac{\partial I(\position, \Omegabf, E, t)}{\partial t} 
+\Omegabf \dotp \bnabla I(\position, \Omegabf, E, t)
+\sigma_t(\position,E) I(\position, \Omegabf, E, t) \\
=& \frac{\sigma_s(\position,E)}{4\pi} \phi(E)
+ \sigma_a(\position,E) B(E,T(\position, t))\\
&+
\biggr[
\biggr( \sigma_t + E \frac{\partial \sigma_a}{\partial E} \biggr) I
+\frac{\sigma_s}{4\pi}
\biggr(
2\phi - E \frac{\partial \phi}{\partial E}
\biggr)
+2\sigma_a B(E,T)
- B(E,T) \frac{\partial \sigma_a}{\partial E}
-\sigma_a E \frac{\partial B(E,T)}{\partial E}
\biggr] \Omegabf \dotp \frac{\mathbf{u}}{c} \\
&-\frac{\sigma_s}{4\pi} \biggr( \RadF - E \frac{\partial \RadF}{\partial E} \biggr) \dotp \frac{\mathbf{u}}{c}
\eeqn 
\newline
\newline
\textbf{Radiation energy equation:}\newline 
Obtained by integrating the transport equation over energy and angle
\beqn 
 \frac{\partial \RadE(\position, t)}{\partial t} 
+\bnabla \dotp \RadF(\position, t) &= \int_0^\infty \sigma_a(\position, E) \bigr( 4\pi B(E,T) - \phi(\position, E, t) \bigr) dE \\
&+\int_0^\infty \biggr( \sigma_a + E \frac{\partial \sigma_a}{\partial E} - \sigma_s(E)\biggr)
\RadF \dotp  \frac{\mathbf{u}}{c}
dE
\eeqn 
\newline
\newline 
\textbf{Radiation momentum equation:}\newline
Obtained by first multiplying by $\frac{1}{c} \Omegabf$, then integrating over all directions and energies,
\beqn 
\frac{1}{c^2} \partialderiv{\RadF}{t} + \bnabla \dotp \RadP &= -\int_0^\infty \frac{\sigma_t}{c} \RadF dE \\
&+\int_0^\infty \bigr( \sigma_s \phi + \sigma_a 4\pi B(E,T) \bigr) \frac{\mathbf{u}}{c^2}dE \\
&+\int_0^\infty \biggr( \sigma_a + E \partialderiv{\sigma_a}{E} + \sigma_s\biggr) \RadP \dotp \frac{\mathbf{u}}{c} dE
\eeqn 



\subsection{Grey Radiative Transfer}
\beqn \label{eq:grey_radiative_transfer}
&\frac{1}{c} \frac{\partial I(\position, \Omegabf, t)}{\partial t} 
+\Omegabf \dotp \bnabla I(\position, \Omegabf, t)
+\sigma_t(\position) I(\position, \Omegabf, t) \\
=& \frac{\sigma_s}{4\pi} \phi
+ \frac{\sigma_a}{4\pi} a c T^4\\
&+
\biggr[
\sigma_t  I
+\frac{\sigma_s}{4\pi} 2\phi 
+2\sigma_a \frac{1}{4\pi} acT^4
-\sigma_a E \frac{\partial B(E,T)}{\partial E}
\biggr] \Omegabf \dotp \frac{\mathbf{u}}{c} \\
&-\frac{\sigma_s}{4\pi}  \RadF \dotp \frac{\mathbf{u}}{c}
\eeqn 
\textbf{Radiation energy equation:}\newline 
Obtained by integrating Eq. \eqref{eq:grey_radiative_transfer} over energy and angle
\beqn \label{eq:radiation_energy_equation}
\frac{\partial \RadE(\position, t)}{\partial t} 
+\bnabla \dotp \RadF(\position, t) &=  \sigma_a c \bigr( aT^4 - \RadE \bigr)
+ \bigr( \sigma_a  - \sigma_s\bigr)
\RadF \dotp  \frac{\mathbf{u}}{c}
\eeqn 
\newline
\newline 
\textbf{Radiation momentum equation:}\newline
Obtained by first multiplying Eq. \eqref{eq:grey_radiative_transfer} by $\frac{1}{c} \Omegabf$, then integrating over all directions and energies,
\beqn \label{eq:radiation_momentum_equation}
\frac{1}{c^2} \partialderiv{\RadF}{t} + \bnabla \dotp \RadP &= - \frac{\sigma_t}{c} \RadF  
+ \bigr( \sigma_s c\RadE + \sigma_a acT^4 \bigr) \frac{\mathbf{u}}{c^2} 
+\sigma_t \RadP \dotp \frac{\mathbf{u}}{c}
\eeqn 

\subsection{Grey Diffusion Approximation}
Approximating the angular dependence of $I(\Omegabf)$ with a $P_1$ spherical harmonic expansion, such that the entries of $\RadP$ are given by
\beqn 
(\RadP)_{i,j}= \frac{1}{3} \RadE \delta_{i,j},
\eeqn 
the radiation energy equation is unaffected but the radiation momentum equation changes. We repeat the radiation energy equation below, and the altered radiation moment equations:
\beqn 
\frac{\partial \RadE}{\partial t} 
+\bnabla \dotp \RadF(\position, t) &=  \sigma_a c \bigr( aT^4 - \RadE \bigr)
+ \bigr( \sigma_a  - \sigma_s\bigr)
\RadF \dotp  \frac{\mathbf{u}}{c},
\eeqn 
\beqn 
\frac{1}{3} \bnabla \RadE = - \frac{\sigma_t}{c} \RadF  
+ \bigr( \sigma_s c\RadE + \sigma_a acT^4 \bigr) \frac{\mathbf{u}}{c^2} 
+\sigma_t \frac{1}{3} \RadE \frac{\mathbf{u}}{c}.
\eeqn 
\newline
\newline 
\textbf{Useful transformations:}\newline 
\begin{subequations}
\beqn 
\RadE_0 = \RadE - \frac{2}{c^2} \RadF \dotp \mathbf{u}
\eeqn 
\beqn 
\RadE = \RadE_0 + \frac{2}{c^2} \RadF_0 \dotp \mathbf{u}
\eeqn 
\beqn \label{eq:F_0_raw}
\RadF_0 = \RadF - \bigr( \RadE \mathbf{u} + \RadP \dotp \mathbf{u} \bigr)
\eeqn 
\beqn 
\RadF = \RadF_0 + \bigr( \RadE_0 \mathbf{u} + \RadP_0 \dotp \mathbf{u} \bigr)
\eeqn 
\beqn 
\RadP_0 = \RadP - \frac{2}{c^2} \mathbf{u} \otimes \RadF
\eeqn 
\beqn 
\RadP = \RadP_0 + \frac{2}{c^2} \mathbf{u} \otimes \RadF_0
\eeqn 
With the $P_1$ approximation
\beqn \label{eq:F0_semi_raw}
\RadF_0 = \RadF - \frac{4}{3} \RadE \mathbf{u}
\eeqn 
\beqn \label{eq:F_semi_raw}
\RadF = \RadF_0 + \frac{4}{3} \RadE \mathbf{u}
\eeqn 
\end{subequations}
Applying these transformations the radiation energy- and moment equation can be expressed as
\beqn 
\frac{\partial \RadE}{\partial t} 
+\bnabla \dotp \RadF(\position, t) &=  \sigma_a c \bigr( aT^4 - \RadE_0 \bigr)
-\sigma_t \RadF \dotp  \frac{\mathbf{u}}{c},
\eeqn 
\beqn 
\frac{1}{3} \bnabla \RadE = - \frac{\sigma_t}{c} \RadF_0
+ \sigma_a c\bigr( aT^4 - \RadE \bigr) \frac{\mathbf{u}}{c^2}.
\eeqn 
Several simplifications to these equations are made. Firstly arriving at the expression for the radiation energy equation,
\beqn 
\frac{\partial \RadE}{\partial t} 
+\bnabla \dotp \RadF(\position, t) &=  \sigma_a c \bigr( aT^4 - \RadE \bigr)
-\sigma_t \RadF_0 \dotp  \frac{\mathbf{u}}{c},
\eeqn 
then the radiation momentum equation,
\beqn 
\frac{1}{3} \bnabla \RadE &= - \frac{\sigma_t}{c} \RadF_0 
\eeqn 
from which we can get expression for $\RadF_0$ and $\RadF$ in terms of $\RadE$ as
\beqn 
\RadF_0 &= -\frac{c}{3\sigma_t} \bnabla \RadE \\
\eeqn 
and
\beqn 
\frac{1}{3} \bnabla \RadE &= - \frac{\sigma_t}{c} \biggr (\RadF - \frac{4}{3} \RadE \mathbf{u} \biggr) \\
\therefore
 \RadF  &= -\frac{c}{3\sigma_t} \bnabla \RadE + \frac{4}{3} \RadE \mathbf{u}.
\eeqn 
These expressions for $\RadF_0$ and $\RadF$ are both then inserted into the radiation energy equation as follows
\beqn 
\frac{\partial \RadE}{\partial t} 
+\bnabla \dotp \RadF(\position, t) &=  \sigma_a c \bigr( aT^4 - \RadE \bigr)
-\sigma_t \RadF_0 \dotp  \frac{\mathbf{u}}{c}
\\
\to
\frac{\partial \RadE}{\partial t} 
+\bnabla \dotp \biggr(  -\frac{c}{3\sigma_t} \bnabla \RadE + \frac{4}{3} \RadE \mathbf{u}  \biggr)
&=  \sigma_a c \bigr( aT^4 - \RadE \bigr)
-\sigma_t \biggr( -\frac{c}{3\sigma_t} \bnabla \RadE \biggr) \dotp  \frac{\mathbf{u}}{c} 
\\
\to 
\frac{\partial \RadE}{\partial t} 
+\bnabla \dotp \biggr(  -\frac{c}{3\sigma_t} \bnabla \RadE \biggr) + \frac{4}{3} \bnabla  \dotp \bigr( \RadE \mathbf{u}  \bigr)
&=  \sigma_a c \bigr( aT^4 - \RadE \bigr)
+\frac{1}{3} \bnabla \RadE  \dotp  \velocity.
\eeqn 

Arriving at a \textbf{diffusion form} of the \textbf{radiation energy equation},
\beqn 
\frac{\partial \RadE}{\partial t} 
+\bnabla \dotp \biggr(  -\frac{c}{3\sigma_t} \bnabla \RadE \biggr) + \frac{4}{3} \bnabla \dotp \bigr( \RadE \mathbf{u}  \bigr)
&=  \sigma_a c \bigr( aT^4 - \RadE \bigr)
+\frac{1}{3} \bnabla \RadE  \dotp  \velocity.
\eeqn

\vspace{1cm}
\subsection{Conservation equation for fluid flow}
The governing equations we consider here are the Euler equations defined as
\beqn 
\partialderiv{\rho}{t} + \bnabla \dotp (\rho \velocity) = 0
\eeqn 
\beqn 
\partialderiv{(\rho\velocity)}{t} + \bnabla \dotp \{ \rho \velocity \otimes \velocity\}  + \bnabla p = \mathbf{f},
\eeqn 
\beqn 
\partialderiv{E}{t} + \bnabla \dotp [(E + p)\velocity] = q
\eeqn 
where $\rho$ is the fluid density, $\velocity = [u_x, u_y, u_z] =[u,v,w]$ is the fluid velocity in cartesian coordinates, $p$ is the fluid pressure, $E$ is the material energy-density comprising kinetic energy-density, $\frac{1}{2} \rho ||\velocity||^2$, and internal energy-density, $\rho e$, such that $E = \frac{1}{2} \rho ||\velocity||^2 + \rho e$, where $e$ is the specific internal energy. The values $q$ and $\mathbf{f}$ are abstractly used here as energy- and moment- sources/sinks, respectively.

The ideal gas law provides the closure relation
\beqn 
p = (\gamma - 1) \rho e
\eeqn 
where $\gamma$ is the ratio of the constant-pressure specific heat, $c_p$, to the constant-volume specific heat, $c_v$, i.e., $\gamma = \frac{c_p}{c_v}$, and is a material property.
\newline
\newline
\textbf{Coupling terms:}\newline
\beqn 
\mathbf{f} &= \frac{\sigma_t}{c} \RadF_0 \\ &= -\frac{1}{3} \bnabla \RadE
\eeqn 
and
\beqn 
q &= - \biggr(\sigma_a c (a T^4 - \RadE) - \sigma_t \RadF_0 \dotp \frac{\mathbf{u}}{c} \biggr)\\
&= \sigma_a c (\RadE - a T^4) - \frac{1}{3} \bnabla \RadE \dotp \velocity
\eeqn 

\newpage
\subsection{The set of Radiation Hydrodynamics Grey Diffusion Equations}
\begin{subequations}
\beqn 
\partialderiv{\rho}{t} + \bnabla \dotp (\rho \velocity) = 0
\eeqn 
\beqn 
\partialderiv{(\rho\velocity)}{t} + \bnabla \dotp \{ \rho \velocity \otimes \velocity\}  + \bnabla p 
= -\frac{1}{3} \bnabla \RadE,
\eeqn 
\beqn 
\partialderiv{E}{t} + \bnabla \dotp [(E + p)\velocity] 
= \sigma_a c (\RadE - a T^4) - \frac{1}{3} \bnabla \RadE \dotp \velocity
\eeqn 
\beqn 
\frac{\partial \RadE}{\partial t} 
+\bnabla \dotp \biggr(  -\frac{c}{3\sigma_t} \bnabla \RadE \biggr) + \frac{4}{3} \bnabla \bigr( \RadE \mathbf{u}  \bigr)
&=  \sigma_a c \bigr( aT^4 - \RadE \bigr)
+\frac{1}{3} \bnabla \RadE  \dotp \velocity.
\eeqn
where
\beqn 
E = \frac{1}{2} \rho ||\velocity||^2 + \rho e,
\eeqn 
\beqn 
p = (\gamma - 1) \rho e,
\eeqn 
\beqn 
T = \frac{1}{C_v} e
\eeqn 
\end{subequations}



\section{Notations}
First we define the following terms
\begin{subequations}
\begin{itemize}
\item The radiation momentum source
\beqn 
\mathbf{S}_{rp} = \frac{1}{3} \bnabla \RadE
\eeqn 
\item The radiation energy source
\beqn 
S_{re} = \sigma_a c \bigr( aT^4 - \RadE \bigr)
+\frac{1}{3} \bnabla \RadE  \dotp \velocity
\eeqn 
\item The conserved hydrodynamic variables
\beqn
\HydroU = 
\begin{bmatrix}
	\rho \\ \rho \mathbf{u} \\ E
\end{bmatrix} 
\eeqn
\item The hydrodynamic flux
\beqn 
\HydroF = 
\begin{bmatrix}
	\rho u \\
	\rho uu + p \\
	\rho u v \\
	\rho u w \\
	(E+p)u
\end{bmatrix}
\eeqn 
\item The radiation energy current
\beqn 
\mathbf{J} = -\frac{c}{3\sigma_t} \bnabla \RadE
\eeqn 
\end{itemize}
\end{subequations}

\noindent
Next, we use these terms to define a more condensed version of the RHGD equations. 
\beqn 
\partialderiv{\HydroU}{t} + \bnabla \dotp \HydroF(\HydroU) = 
\begin{bmatrix}
	0 \\
-\mathbf{S}_{rp} \\
-S_{re} 
\end{bmatrix}
\eeqn 
\beqn 
\frac{\partial \RadE}{\partial t} 
+\bnabla \dotp \RadJ  + \frac{4}{3} \bnabla \dotp \bigr( \RadE \mathbf{u}  \bigr)
&=  S_{re}.
\eeqn

\newpage
\section{Overview of temporal numerical scheme}
\begin{subequations}
\beqn 
\partialderiv{\HydroU}{t} + \bnabla \dotp \HydroF(\HydroU) = 
\begin{bmatrix}
	0 \\
	-\mathbf{S}_{rp} \\
	-S_{re} 
\end{bmatrix}
\eeqn 
\beqn 
\frac{\partial \RadE}{\partial t} 
+\bnabla \dotp \RadJ  + \frac{4}{3} \bnabla \dotp \bigr( \RadE \mathbf{u}  \bigr)
&=  S_{re}.
\eeqn
\end{subequations}



\subsection{Predictor phase}
$\tau = \dfrac{1}{\half \Delta t}$
\begin{subequations}
	\beqn 
	\tau (\HydroU^{n+\half*} - \HydroU^{n}) + \bnabla \dotp \HydroF(\HydroU^{n}) = \mathbf{0}
	\eeqn 
	\beqn 
	\tau (\RadE^{n+\half*} - \RadE^{n}) + \biggr(\frac{4}{3} \bnabla \dotp \bigr(\RadE \velocity)\biggr)^{n} = 0
	\eeqn 
	\beqn 
	\tau (\HydroU^{n+\half} - \HydroU^{n+\half*})_{0,1} =  \begin{bmatrix}
		0 \\
		-\mathbf{S}_{rp} 
	\end{bmatrix}^{n}
	\eeqn 
	\beqn 
	\tau (\HydroU^{n+\half} - \HydroU^{n+\half*})_{0} = 
	\half \sigma_a^{n+\half} c  \biggr( 
	\RadE^{n+\half} + \RadE^{n}
	-a\bigr( T^{4,n+\half} + T^{4,n} \bigr)
	\biggr)
	- \biggr(\frac{1}{3} \bnabla \RadE \dotp \velocity \biggr)^{n}
	\eeqn 
	\beqn 
	\tau (\RadE^{n+\half} - \RadE^{n+\half*}) 
	+ \half \bnabla \dotp \bigr( \RadJ^{n+\half} +  \RadJ^{n} \bigr)= 
	\half \sigma_a^{n+\half} c \biggr( 
	a\bigr( T^{4,n+\half} + T^{4,n} \bigr)  -\RadE^{n+\half} - \RadE^{n}
	\biggr)
	+ \biggr( \frac{1}{3} \bnabla \RadE \dotp \velocity \biggr)^{n}
	\eeqn
	
	
	\beqn 
	T^{4,n+\half} = T^{4,n+\half*} + \frac{4T^{3,n+\half*}}{C_v} (e^{n+\half}-e^{n+\half*})
	\eeqn 
\end{subequations}

\newpage
\subsection{Corrector phase}
$\tau = \dfrac{1}{\Delta t}$
\begin{subequations}
	\beqn 
	\tau (\HydroU^{n+\half*} - \HydroU^{n}) + \bnabla \dotp \HydroF(\HydroU^{n+\half}) = \mathbf{0}
	\eeqn 
	\beqn 
	\tau (\HydroU^{n+1} - \HydroU^{n+\half*})_{0,1} =  \begin{bmatrix}
		0 \\
		-\mathbf{S}_{rp} 
	\end{bmatrix}^{n+\half}
	\eeqn 
	\beqn 
	\tau (\RadE^{n+\half*} - \RadE^{n}) + \biggr(\frac{4}{3} \bnabla \dotp \bigr(\RadE \velocity)\biggr)^{n+\half} = 0
	\eeqn 
	\beqn 
	\tau (\HydroU^{n+1} - \HydroU^{n+\half*})_{0} = 
	\half \sigma_a^{n+1} c  \biggr( 
	\RadE^{n+1} + \RadE^{n}
	-a\bigr( T^{4,n+1} + T^{4,n} \bigr)
	\biggr)
	- \biggr(\frac{1}{3} \bnabla \RadE \dotp \velocity \biggr)^{n+\half}
	\eeqn 
	\beqn 
	\tau (\RadE^{n+1} - \RadE^{n+\half*}) 
	+ \half \bnabla \dotp \bigr( \RadJ^{n+1} +  \RadJ^{n} \bigr)= 
	\half \sigma_a^{n+1} c \biggr( 
	a\bigr( T^{4,n+1} + T^{4,n} \bigr)  -\RadE^{n+1} - \RadE^{n}
	\biggr)
	+ \biggr( \frac{1}{3} \bnabla \RadE \dotp \velocity \biggr)^{n+\half}
	\eeqn
	
	
	\beqn 
	T^{4,n+1} = T^{4,n+\half*} + \frac{4T^{3,n+\half*}}{C_v} (e^{n+1}-e^{n+\half*})
	\eeqn 
\end{subequations}







\newpage
\section{Finite Volume Spatial Discretization}
We now integrate $\mathbf{S}_{rp}$, $S_{re}$, the hydrodynamic equations and the radiation energy equation over the volume of a cell with index $c$ and volume $V_c$ to get
\beqn 
V_c \mathbf{S}_{rp} = \frac{1}{3} \SumOverCellFaces \biggr( \AreaVec_f \RadE_f \biggr)
\eeqn 
\beqn 
V_c \mathbf{S}_{re} = V_c \sigma_a c (a T_c^4 - \RadE_c) 
+ \frac{1}{3} \SumOverCellFaces \biggr(\AreaVec_f \dotp \RadE_f \velocity_f \biggr)
\eeqn 
\beqn 
V_c\partialderiv{\HydroU}{t} + \SumOverCellFaces \biggr( \AreaVec_f \dotp \HydroF_f \biggr) = 
\begin{bmatrix}
	0 \\
	-V_c \mathbf{S}_{rp} \\
	-V_c S_{re} 
\end{bmatrix}
\eeqn 
\beqn 
\frac{\partial \RadE}{\partial t} 
+\SumOverCellFaces \biggr( \AreaVec_f \dotp \RadJ_f \biggr) + \frac{4}{3} \SumOverCellFaces \biggr( \AreaVec_f \dotp \RadE_f \velocity_f \biggr) = V_c S_{re}\\
\eeqn
where $\AreaVec_f$ denotes the area-vector $A_f \mathbf{n}$ and all quantities with an $f$ subscript as of yet unresolved. 






\newpage
\begin{appendices}
\section{Angular integration identities} \label{appendix:angle_integration_identities}
\begin{description}
	\item [Identity A-1]
	$$\int_{4\pi}  \ d\Omegabf = 4\pi.$$
	
	\item [Identity A-2]
	$$\int_{4\pi} \Omegabf \ d\Omegabf = 0.$$
	
	\item [Identity A-3] Given the known three component vector, $\mathbf{v}$,
	$$\int_{4\pi} \Omegabf \dotp \mathbf{v} \ d\Omegabf = 0.$$
	
	\item [Identity A-4] Given the known three component vector, $\mathbf{v}$,
	$$\int_{4\pi} \Omegabf \dotp \bnabla (\Omegabf \dotp \mathbf{v}) \ d\Omegabf = \frac{4\pi}{3} \bnabla \dotp \mathbf{v}.$$
	
	\item [Identity A-5] Given the scalar, $a$,
	$$\int_{4\pi} \Omegabf \biggr( \Omegabf \dotp \bnabla a \biggr) \ d\Omegabf = \frac{4\pi}{3} \bnabla a.$$
	
	\item [Identity A-6] Given the known three component vector, $\mathbf{v}$,
	$$\int_{4\pi} \Omegabf \biggr( \Omegabf \dotp \mathbf{v} \biggr) \ d\Omegabf = \frac{4\pi}{3} \mathbf{v} .$$
	
	\item [Identity A-7] Given the known three component vector, $\mathbf{v}$,
	$$\int_{4\pi} \Omegabf \biggr( \Omegabf \dotp \bnabla (\Omegabf \dotp \mathbf{v}) \biggr) \ d\Omegabf = 0 .$$
	
\end{description}

\end{appendices}


\newpage
\begin{thebibliography}{1}
	
%	\bibitem{LewisMiller} Lewis E.E., Miller W.F., {\em Computational Methods of Neutron Transport}, JohnWiley \& Sons, 1984

	\bibitem{Castor} John I. Castor, {\em Radiation Hydrodynamics}, UCRL-BOOK-134209-Rev-1, Lawrence Livermore National Laboratory, November 2003.
	
	\bibitem{Toro} Toro E.F., {\em Riemann Solvers and Numerical Methods for Fluid Dynamics - A Practical Introduction}, third edition, Springer, 2009.
	
	\bibitem{Moukalled} Moukalled F.,  Mangani L., Darwish M., {\em The Finite Volume Method in Computational Fluid Dynamics - An Advanced Introduction with OpenFOAM® and Matlab®}, Springer, 2016.
	
	\bibitem{McClarrenSlopes} McClarren R.G., Lowrie R.B., {\em The effects of slope limiting on asymptotic-preserving numerical methods for hyperbolic conservation laws}, Journal of Computational Physics, vol 227 p9711-9726, 2008.
	
	\bibitem{Timmers} Timmers F.X., {\em Exact Riemann Solver}, Website: https://cococubed.com/code\_pages/exact\_riemann.shtml, accessed April 22, 2022.
	
	   
\end{thebibliography}

\newpage
\begin{appendices}
\section{Roderigues's formula} \label{appendix:Roderigues_formula}
Roderigues' formula for the rotation of a vector $\mathbf{v}$ about a unit vector $\mathbf{a}$ with right-hand rule
\begin{equation}
\newcommand{\vvec}{\mathbf{v}}
\newcommand{\avec}{\mathbf{a}}
\begin{aligned}
\vvec_{rotated} &= \cos \theta \vvec + (\avec \dotp \vvec)(1-\cos \theta) \avec + \sin \theta (\avec \times \vvec)
\end{aligned}
\end{equation}
In matrix form
\beqn 
\mathbf{v}_{rotated} = A \mathbf{v}
\eeqn 
where
\beqn 
A = 
\begin{bmatrix}
0 & -a_z & a_y \\
a_z & 0 & -a_x \\
-a_y & a_x & 0
\end{bmatrix}
\eeqn 
and
\beqn 
R = I + \sin\theta A + (1-\cos\theta) A^2
\eeqn


\end{appendices}

\end{document}