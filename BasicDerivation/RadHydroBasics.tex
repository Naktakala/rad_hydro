\documentclass[10pt,letterpaper,notitlepage]{article}
\usepackage[utf8]{inputenc}
\usepackage{amsmath}
\usepackage{amsfonts}
\usepackage{amssymb}
\usepackage{graphicx}
\usepackage{cancel}
\usepackage{float}
\usepackage{cite}

\usepackage[ruled,vlined]{algorithm2e}


\usepackage[left=0.75in, right=0.75in, bottom=1.0in,top=0.75in]{geometry}

%\usepackage{caption} 
%\captionsetup[table]{skip=10pt}
%\usepackage[font=small,labelfont=bf]{caption}

\usepackage{comment}
\usepackage{listings}

\usepackage{color}
\definecolor{Brown}{cmyk}{0,0.81,1,0.60}
\definecolor{OliveGreen}{cmyk}{0.64,0,0.95,0.40}
\definecolor{CadetBlue}{cmyk}{0.62,0.57,0.23,0}

\usepackage{multicol}

\usepackage{appendix}

\usepackage{fancyhdr}
%\usepackage[colorlinks=true,linkcolor=blue,urlcolor=black,bookmarksopen=true,bookmarks]{hyperref}
\usepackage{bookmark}


%============================= Put document title here
\newcommand{\DOCTITLE}{Basic derivations of Radiation-Hydrodynamics}  

%=============================  Load list of user-defined commands
% Mark URL's
\newcommand{\URL}[1]{{\textcolor{blue}{#1}}}
%
% Ways of grouping things
%
\newcommand{\bracket}[1]{\left[ #1 \right]}
\newcommand{\bracet}[1]{\left\{ #1 \right\}}
\newcommand{\fn}[1]{\left( #1 \right)}
\newcommand{\ave}[1]{\left\langle #1 \right\rangle}
\newcommand{\norm}[1]{\Arrowvert #1 \Arrowvert}
\newcommand{\abs}[1]{\arrowvert #1 \arrowvert}
%
% Bold quantities
% 
\newcommand{\Omegabf}{\mathbf{\Omega}}
\newcommand{\bnabla}{\boldsymbol{\nabla}}
\newcommand{\position}{\mathbf{x}}
\newcommand{\dotp}{\boldsymbol{\cdot}}
%
% Vector forms
%
\renewcommand{\vec}[1]{\mbox{$\stackrel{\longrightarrow}{#1}$}}
\renewcommand{\div}{\mbox{$\vec{\mathbf{\nabla}} \cdot$}}
\newcommand{\grad}{\mbox{$\vec{\mathbf{\nabla}}$}}
\newcommand{\bb}[1]{\bar{\bar{#1}}}
%
% Vector forms boldfaced
\newcommand{\bvec}[1]{\mathbf{#1}}
\newcommand{\bdiv}{\boldsymbol{\nabla} \boldsymbol{\cdot}}
\newcommand{\bgrad}{\bnabla}
\newcommand{\mat}[1]{\bar{\bar{#1}}}
%
%
% Equation beginnings and endings
%
% Un-numbered equation with alignment
\newcommand{\beq}{\begin{equation*} \begin{aligned}}
\newcommand{\eeq}{\end{aligned}\end{equation*}}
% Numbered equation with alignment
\newcommand{\beqn}{\begin{equation}\begin{aligned}}
\newcommand{\eeqn}{\end{aligned}\end{equation}}  

%
% Quick commands for symbols
%
\newcommand{\Edensity}{\mathcal{E}}


\newcommand{\jcr}[1]{\textcolor{magenta}{#1}}
\usepackage[normalem]{ulem}
\newcommand{\ssout}[1]{\sout{\textcolor{magenta}{#1}}}

%
% Code syntax highlighting
%
%\lstset{language=C++,frame=ltrb,framesep=2pt,basicstyle=\linespread{0.8} \small,
%	keywordstyle=\ttfamily\color{OliveGreen},
%	identifierstyle=\ttfamily\color{CadetBlue}\bfseries,
%	commentstyle=\color{Brown},
%	stringstyle=\ttfamily,
%	showstringspaces=true,
%	tabsize=2,}

\lstset{language=C++,frame=ltrb,framesep=8pt,basicstyle=\linespread{0.8} \Large,
commentstyle=\ttfamily\color{OliveGreen},
keywordstyle=\ttfamily\color{blue},
identifierstyle=\ttfamily\color{CadetBlue}\bfseries,
stringstyle=\ttfamily,
tabsize=2,
showstringspaces=false,
numbers=left,
captionpos=t}

\renewcommand{\lstlistingname}{\textbf{Code Snippet}}% Listing -> Code Snippet


\begin{document}
\noindent
{\LARGE\textbf{\DOCTITLE}}
\newline
\newline
\newline
\noindent
{\Large Jan I.C. Vermaak$^{1,2}$, Jim E. Morel$^{1,2}$}
\newline
\noindent\rule{\textwidth}{1pt}
{\small $^1$Center for Large Scale Scientific Simulations, Texas A\&M Engineering Experiment Station, College Station, Texas, USA.}
\newline\noindent
{\small $^2$Nuclear Engineering Department, Texas A\&M University, College Station, Texas, USA.}
\newline
\newline
\textbf{Abstract:}\newline\noindent
Work is work for some, but for some it is play.
\newline
\newline\noindent
{\small
\textbf{Keywords:} transport sweeps; discrete-ordinate method; radiation transport; massively parallel simulations; discontinuous Galerkin; unstructured mesh}

\section{Definitions}
\subsection{Independent variables}
We refer to the following independent variables:
\begin{itemize}
\item Position in the cartesian space $\{x,y,z\}$ is denoted with $\position$ and each component having units $[cm]$.
\item Direction, $\{\varphi, \theta\}$, is denoted with $\Omegabf$ which takes on the form 
$$
\Omegabf = 
\begin{bmatrix}
\Omega_x \\ \Omega_y \\ \Omega_z
\end{bmatrix}
\text{ and/or }
\Omegabf = 
\begin{bmatrix}
\sin\theta \cos\varphi \\ \sin\theta \sin\varphi \\ \cos\theta
\end{bmatrix},
$$
where $\varphi$ is the azimuthal-angle and $\theta$ is the polar-angle, both in spherical coordinates. Commonly, $\cos\theta$, is denoted with $\mu$. The general dimension of angular phase space is $[steridian]$.
\item Photon frequency, $\nu$ in $[Hertz]$ or $[s^{-1}]$.
\item Time, $t$ in $[s]$.
\end{itemize} 

\vspace{0.5cm}
\subsection{Dependent variables}
We use the following basic dependent variables:
\begin{itemize}
\item The foundation of the dependent unknowns is the \textbf{radiation angular intensity}, $I(\position, \Omegabf, \nu,t)$ with units $[Joule/cm^2 {-} s {-} steradian {-} Hz]$. We often use the corresponding angle-integral of this quantity, $\phi(\position,\nu,t)$, and define it as
\beqn 
\phi(\position,\nu,t) = \Edensity c = \int_{4\pi} I(\position,\Omegabf,\nu,t) \ d\Omegabf
\eeqn 
with units $[Joule/cm^2 {-} s {-} Hz]$.
\item  The \textbf{radiation energy density}, $\Edensity$, is 
\beqn 
\Edensity(\position, \nu, t) = \frac{\phi}{c}  = 
\frac{1}{c} \int_{4\pi} I(\position,\Omegabf,\nu,t) \ d\Omegabf
\eeqn 
with units $[Joule/cm^3 {-} Hz]$.
\end{itemize}


\vspace{0.5cm}
\subsection{Blackbody radiation}
A blackbody source, $B(\nu,T)$, is properly described by Planck's law,
\beqn \label{eq:plancks_law}
B(\nu,T) = \frac{1}{4\pi}\frac{2h\nu^3}{c^2} \frac{1}{e^{\frac{h\nu}{k_B T}} - 1  }
\eeqn 
with units $[Joule/cm^2 {-} s {-} steridian-Hz]$ where $h$ is Planck's constant and $k_B$ is the Boltzmann constant.

If we integrate the blackbody source over all angle-space and frequencies the we get the mean radiation intensity from a blackbody at temperature $T$ as
\beqn 
\int_0^\infty \int_{4\pi}  B(\nu, T) \ d\Omegabf d\nu
&=
\int_0^\infty \int_{4\pi}  \frac{1}{4\pi}\frac{2h\nu^3}{c^2} \frac{1}{e^{\frac{h\nu}{k_B T}} - 1  } \ d\Omegabf d\nu \\
&= a c T^4,
\eeqn 
with units $[Joule/cm^2 {-} s {-} steridian]$ and where $a$ is the \textbf{blackbody radiation constant} given by
\beqn 
a = \frac{8\pi^5 k_B^4}{15 h^3 c^3}.
\eeqn 
\newline
\newline
In both cases this unfortunately is only the intensity. Following Kirchoff's law, which states that the emission and absorption of radiation must be equal in equilibrium, we can determine the \textbf{blackbody emission rate}, $S_{bb}$, from the absorption rate as
\beqn 
S_{bb}(\nu, T) = \rho \kappa(\nu) B(\nu, T),
\eeqn 
with units $[Joule/cm^3 {-} s {-} steridian {-} Hz]$
where $\rho$ is the material density $[g/cm^3]$ and $\kappa$ is the opacity $[cm^2/g]$. Data for the opacity of a material is normally available in the form of either the \textbf{Rosseland opacity}, $\kappa_{Rs}$, or the \textbf{Planck opacity}, $\kappa_{Pl}$.






\vspace{1cm}
\section{Conservation equation - Electromagnetic Radiation}
The basic statement of conservation, without hydrodynamics, is
\beqn 
\frac{1}{c} \frac{\partial I(\position, \Omegabf, \nu, t)}{\partial t} &=
-\Omegabf \dotp \bnabla I(\position, \Omegabf, \nu, t)
- \sigma_t(\position,\nu) I(\position, \Omegabf, \nu, t) \\
&+ \int_0^{\infty} \int_{4\pi} \frac{\nu}{\nu'} \sigma_s(\position,\nu'{\to}\nu,\Omegabf'{\dotp}\Omegabf) I(\position, \Omegabf', 
\nu, t)  d\nu' d\Omegabf'
+S
\eeqn 

\newpage
\begin{thebibliography}{1}
	
	\bibitem{LewisMiller} Lewis E.E., Miller W.F., {\em Computational Methods of Neutron Transport}, JohnWiley \& Sons, 1984
	   
\end{thebibliography}

\newpage
\begin{appendices}
\section{First appendix}
Put ``Lazy reader stuff here".
\end{appendices}

\end{document}